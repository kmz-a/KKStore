%%
%% This is file `tikzposter-template.tex',
%% generated with the docstrip utility.
%%
%% The original source files were:
%%
%% tikzposter.dtx  (with options: `tikzposter-template.tex')
%%
%% This is a generated file.
%%
%% Copyright (C) 2014 by Pascal Richter, Elena Botoeva, Richard Barnard, and Dirk Surmann
%%
%% This file may be distributed and/or modified under the
%% conditions of the LaTeX Project Public License, either
%% version 2.0 of this license or (at your option) any later
%% version. The latest version of this license is in:
%%
%% http://www.latex-project.org/lppl.txt
%%
%% and version 2.0 or later is part of all distributions of
%% LaTeX version 2013/12/01 or later.
%%


\documentclass{tikzposter} %Options for format can be included here

\usepackage{todonotes}

\usepackage[tikz]{bclogo}
\usepackage{lipsum}
\usepackage{amsmath}

\usepackage{booktabs}
\usepackage{longtable}
\usepackage[absolute]{textpos}
\usepackage[it]{subfigure}
\usepackage{graphicx}
\usepackage{cmbright}
%\usepackage[default]{cantarell}
%\usepackage{avant}
%\usepackage[math]{iwona}
\usepackage[math]{kurier}
\usepackage[T1]{fontenc}


%% add your packages here
\usepackage{hyperref}
% for random text
\usepackage{lipsum}
\usepackage[english]{babel}
\usepackage[pangram]{blindtext}

\colorlet{backgroundcolor}{blue!10}

 % Title, Author, Institute
\title{Predict Future Sale}
\author{MingZhu Kang}
\institute{$^1$ Xi'an Shiyou University, China \\}
%\titlegraphic{logos/tulip-logo.eps}

%Choose Layout
\usetheme{Wave}

%\definebackgroundstyle{samplebackgroundstyle}{
%\draw[inner sep=0pt, line width=0pt, color=red, fill=backgroundcolor!30!black]
%(bottomleft) rectangle (topright);
%}
%
%\colorlet{backgroundcolor}{blue!10}

\begin{document}


\colorlet{blocktitlebgcolor}{blue!23}

 % Title block with title, author, logo, etc.
\maketitle

\begin{columns}
 % FIRST column
\column{0.5}% Width set relative to text width

%%%%%%%%%% -------------------------------------------------------------------- %%%%%%%%%%
 %\block{Main Objectives}{
%  	      	\begin{enumerate}
%  	      	\item Formalise research problem by extending \emph{outlying aspects mining}
%  	      	\item Proposed \emph{GOAM} algorithm is to solve research problem
%  	      	\item Utilise pruning strategies to reduce time complexity
%  	      	\end{enumerate}
%%  	      \end{minipage}
%}
%%%%%%%%%% -------------------------------------------------------------------- %%%%%%%%%%


%%%%%%%%%% -------------------------------------------------------------------- %%%%%%%%%%
\block{Introduction}{
    % Many real world applications call for one important function
    % of identifying the set of features
    % on which the interested object is most distinguished from others.
    % Usually,
    % this object is termed as the query object,
    % and the set of features are referred to as the \emph{subspaces} or \emph{aspects}.
    % Accordingly,
    % this research problem is referred to as
    % \emph{outlying aspects mining},
    % which is different from \emph{outlier detection}.
    In deep learning projects, data visualization operations often need to be carried out, 
    including visualization of original image data, visualization of loss and accuracy, etc.\\
    In this study, the data in the data set are collated and analyzed to analyze 
    the impact of commodity sales volume, product category, operating income 
    and the economic scale of the city where the stores are located on the income.\\
    The model is used to train the data set, and the historical sales data
     is used as the model feature to realize the commodity sales forecast.
     Effectively guide the store to conduct reasonable inventory management.
  	% \begin{description}
  	% \item[Outlying Aspects Mining] aims to identify a subspace
    % which makes the query object most outlying,
    % rather than verifying whether it is an outlier or not.
    % The task of \emph{Outlying Aspects Mining}
    % is to explain which aspects make the query object most different.
  	
  	% \item[Outlier Detection] aims to identify all possible outliers in the dataset,
    % without explaining why or how they are different.
    % Hence,
    % the outlying aspects mining is also referred to
    % \emph{outlier interpretation}
    % or \emph{object explanation}.
  	% \end{description}

  	% In this paper,
    % we extend the task of \emph{outlying aspects mining} to the \emph{group} level,
    % formalize the research problem of \emph{group outlying aspects mining},
    % and propose a novel algorithm named GOAM to solve the
    % \emph{group outlying aspects mining} problem.
}
%%%%%%%%%% -------------------------------------------------------------------- %%%%%%%%%%


%%%%%%%%%% -------------------------------------------------------------------- %%%%%%%%%%
\block{Data Sets}{
\begin{itemize}
    \item Preprocessing of project data sets\\
    % The preprocessing of project data set includes training set, 
    % commodity set, commodity data set, commodity category set 
    % and test set.\\
    % through the data of the training set,
    % we determined the parameters of the fitting curve 
    % to filter the obvious outliers and checked the abnormal conditions of prices
    % and sales with item_CNt_day and item_price.
    The preprocessing of project data set includes training set,
    commodity set, commodity data set, commodity category set
    and test set. \\
    % through the data of the training set,
    % we determined the parameters of the fitting curve 
    % to filter the obvious outliers and checked the abnormal conditions of prices
    % and sales with item_cnt_day and item_price.
    through the data of the training set
    we determined the parameters of the fitting curve
    to filter the obvious outliers and checked the abnormal conditions of prices.
    % of group outlying aspects mining
    % is to explain which aspects make the query group distinctive
    % different  from the other groups.
    \item Training set data cleaning\\
    % \emph{Group Outlying Aspects Mining},
    through the data of the training set,
    we determined the parameters of the fitting curve.\\
    to filter the obvious outliers and checked the abnormal conditions of prices and sales\\ 
    % \emph{Outlying Aspects Mining} and
    % \emph{Outlier Detection} are different with each other.
\end{itemize}

% \begin{center}
%     \begin{minipage}{0.3\linewidth}
%     \centering
%     \begin{tikzfigure}
%     \missingfigure[figcolor=white]{Testing figcolor}
%     {\small{Group Outlying Aspects Mining}}
%     \end{tikzfigure}%
%     \end{minipage}
%     \hfill
%     \begin{minipage}{0.3\linewidth}
%     \centering
%     \begin{tikzfigure}
%     \missingfigure[figcolor=white]{Testing figcolor}
%     {\small{Outlying Aspects Mining}}
%     \end{tikzfigure}%
%     \end{minipage}
%     \hfill
%     \begin{minipage}{0.3\linewidth}
%     \centering
%     \begin{tikzfigure}
%     \missingfigure[figcolor=white]{Testing figcolor}
%     {\small{Outlier Detection}}
%     \end{tikzfigure}%
%     \end{minipage}
% \end{center}
}
%%%%%%%%%% -------------------------------------------------------------------- %%%%%%%%%%


%%%%%%%%%% -------------------------------------------------------------------- %%%%%%%%%%

%\note{Note with default behavior}

%\note[targetoffsetx=12cm, targetoffsety=-1cm, angle=20, rotate=25]
%{Note \\ offset and rotated}

 % First column - second block


%%%%%%%%%% -------------------------------------------------------------------- %%%%%%%%%%
\block{Structured Data And Analysis}{
    \item Sale Analysis\\
    According to the data analysis,
    to the overall sales volume of the store 
    showed a downward trend.
    and the monthly sales volume was mostly lower 
    than the same period of last year
%     \vspace{.5cm}
% \begin{tabular}{ c | c | c }
%     \toprule
%     Teams                               & Pre-ranking algorithm     &  LightGBM    \\
%     \toprule
%     Accumulation of statistic           & O(\#feature\#data)      & O(\#featurek) \\       
%     Direct variance mapping             & N/A                       & Double the speed                        \\
%     Direct support category features    & N/A                       & 8 times faster on Expo data    \\
% %    Golden State Warriors   & \{FG\%\}                  & \{FT\%, Blk\}, \{FGA, 3PT\%, FTA\}\\
% %    Utah Jazz               & \{Blk\}                   & \{3FA, 3PT\%\}                    \\
%     Cache optimization                  & N/A                       & accelerate 40\% on Expo data          \\
%     \bottomrule
% \end{tabular}
% \vspace{.5cm}
    
    \item TurnOver Analysis\\
    Revenue in the 23rd month was up sharply 
    from the 11th month,
    while sales were down year-on-year.
    One item's gross revenue was unusually high.
    \item Product Category Influence\\
    The influence of product category on sales volume
    The first three categories that contribute the most to total sales
     are the 40th, 30th and 55th categories.
     \item Impact Analysis\\
     The impact of the city in which the store is located on total revenue.\\
     Stores in district 14 cities contributed the most to the total revenue.\\
    % According to the data analysis,
    % the overall sales volume of the store 
    % showed a downward trend
    % and the monthly sales volume was mostly lower 
    % than the same period of last year.
    % One item sold exceptionally well.
  	% The GOAM algorithm includes three major steps.
%    1) Group Feature Extraction,
%    2) Outlying Degree Scoring, and
%    3) Outlying Aspects Identification.
  	
% \begin{tikzfigure}%[Overall architecture of \emph{GOAM} algorithm]
% %  \includegraphics[width=0.8\linewidth]{figures//framework.pdf}
%     \missingfigure[figcolor=white]{Testing figcolor}
% \end{tikzfigure}
		
% \begin{description}
%   	\item[Group Feature Extraction]
%   	Let $f_1$, $f_2$, $f_3$ represent three features of $G_q$.
%     We count the frequency of each value for one feature.
%     Then use the histogram to represent each feature.
%     Similarly,
%     we can extract other features for each group.

%    \item
%    The histogram of $G_q$ on three features are as follows.
% \end{description}

% \begin{center}
%     \begin{minipage}{0.3\linewidth}
%     \centering
%     \begin{tikzfigure}
%     \missingfigure[figcolor=white]{Testing figcolor}
%     {\small{Histogram of $G_q$ on $f_1$}}
%     \end{tikzfigure}%
%     \end{minipage}
%     \hfill
%     \begin{minipage}{0.3\linewidth}
%     \centering
%     \begin{tikzfigure}
%     \missingfigure[figcolor=white]{Testing figcolor}
%     {\small{Histogram of $G_q$ on $f_2$}}
%     \end{tikzfigure}%
%     \end{minipage}
%     \hfill
%     \begin{minipage}{0.3\linewidth}
%     \centering
%     \begin{tikzfigure}
%     \missingfigure[figcolor=white]{Testing figcolor}
%     {\small{Histogram of $G_q$ on $f_3$}}
%     \end{tikzfigure}%
%     \end{minipage}
% \end{center}
% \begin{description}
% \item[Outlying Degree Scoring]
%     In this step,
%     we first calculate the \emph{earth mover distance} (EMD) of one feature among different groups.
%     The earth mover distance reflects the minimum mean distance
%     between groups on one feature.
%     So,
%     we utilize the EMD to measure the difference between groups of each feature.
% \end{description}
}
%%%%%%%%%% -------------------------------------------------------------------- %%%%%%%%%%


% SECOND column
\column{0.5}
 %Second column with first block's top edge aligned with with previous column's top.

%%%%%%%%%% -------------------------------------------------------------------- %%%%%%%%%%
\block{Predict Future Trend}{
% \begin{description}
%     \item
%     Second,
%     based on the \emph{earth move distance},
%     we calculate the outlying degree.
% \end{description}

% \begin{tikzfigure}%[Overall architecture of \emph{GOAM} algorithm]
%     \missingfigure[figcolor=white]{Testing figcolor}
% \end{tikzfigure}
  By processing closed stores and goods that are not for sale,
  Keep only stores open for the last six months and items that sell.
  Combining the features of the store data set and the merchandise data set.\\
  The goal is to use historical sales data to predict future sales.
  Using the historical sales data as the characteristics of the model,
  this month's sales results as labels to build a model for regression analysis.


% \begin{description}
%   	\item[Outlying Aspects Identification]
%     In this step,
%     Combining the features of the store data set and the merchandise data set
%     The goal is to use historical sales data to predict future sales.
%     If a feature's outlying degree is greater than a threshold,
%     the more likely the feature is group outlying aspect.
%     When the dimensionality of features is high,
%     we adopt a stage-wise candidate subspace construction strategy to
%     alleviate the exponential explosion.
% \end{description}
}
%%%%%%%%%% -------------------------------------------------------------------- %%%%%%%%%%
% Second column - first block


%%%%%%%%%% -------------------------------------------------------------------- %%%%%%%%%%
\block[titleleft]{Model Training}
{
\begin{description}
    \item[LightGBM Model]
    is a fast, distributed, high-performance 
    gradient enhancement framework based on decision tree algorithms.
    It supports category characteristics.\\
    This project uses lightGBM model for training.
    select lightGBM model for training and combine the predicted results into the test set.
    LightGBM supports category characteristics directly and natively 
    by changing the decision rules of the decision tree algorithm, 
    without transformation.
\end{description}
\vspace{.5cm}
% \begin{tabular}{ c | c | c | c }
%     \toprule
%     Method     &  Truth Outlying Aspects    & Identified Aspects & Accuracy      \\
%     \midrule
%     GOAM       &  $\{F_1\}$, $\{F_2F_4\}$   &  $\{F_1\}$, $\{F_2F_4\}$    & 100\%    \\

%      Arithmetic Mean based OAM &  $\{F_1\}$, $\{F_2F_4\}$   &  $\{F_4\}$, $\{F_2\}$    &  0\% \\

%      Median based OAM &  $\{F_1\}$, $\{F_2F_4\}$   &  $\{F_2\}$, $\{F_4\}$    &           0\% \\
%      \bottomrule
% \end{tabular}
% \vspace{.2cm}
% \begin{description}
%     \item
%     It can be observed that the GOAM method can identify the trivial outlying features
%     and non-trivial outlying subspaces correctly and is obvious from the table
%     that the accuracy of GOAM is the best, which is ($100\%$).
% \end{description}

% \begin{description}
% \item[NBA Dataset] was collected from Yahoo Sports
% website (\url{http://sports.yahoo.com.cn/nba}).
% The data include all teams from the six divisions,
% and each player in the team has $12$ features.
% \end{description}
\vspace{.5cm}

\begin{tabular}{ c | c | c }
    \toprule
    Teams                               & Pre-ranking algorithm     &  LightGBM    \\
    \toprule
    %Accumulation of statistic           & O(\#feature\#data)        & O(\#feature\#data)  
    Segment gain calculation            & O(\#feature\#data)        & O(\#feature) \\       
    Direct variance mapping             & N/A                       & Double the speed                        \\
    Direct support category features    & N/A                       & 8 times faster on Expo data    \\
%    Golden State Warriors   & \{FG\%\}                  & \{FT\%, Blk\}, \{FGA, 3PT\%, FTA\}\\
    Accumulation of statistic           & O(\#feature\#data)        & O(\#feature\#data)                   \\
    Cache optimization                  & N/A                       & accelerate 40\% on Expo data          \\
    \bottomrule
\end{tabular}
        
% \begin{minipage}{0.5\linewidth}
%     \centering
%     \begin{tikzfigure}
%     \missingfigure[figcolor=white]{Testing figcolor}

%     {\small{New Orleans Pelicans on FT\%}}
%     \end{tikzfigure}%
% \end{minipage}
% \hfill
% \begin{minipage}{0.5\linewidth}
%     \centering
%     \begin{tikzfigure}
%     \missingfigure[figcolor=white]{Testing figcolor}

%     {\small{New Orleans Pelicans on FTA}}
%     \end{tikzfigure}%
% \end{minipage}
% \vspace{.2cm}
% \begin{description}
% \item
% \texttt{New Orleans Pelicans} has more players with
% lower \{free throw percentage\}, \{free throws attempted\}.
% \end{description}
}
%%%%%%%%%% -------------------------------------------------------------------- %%%%%%%%%%


% Second column - second block
%%%%%%%%%% -------------------------------------------------------------------- %%%%%%%%%%
\block[titlewidthscale=1, bodywidthscale=1]
{Conclusion}
{
Through this project,
I have learned a lot, including the effective aspects of problem cutting,
code implementation of analysis algorithm, design of analysis process, etc.\\
which enables me to better grasp the thinking of data analysis on the whole.
From data analysis methods to feature engineering and prediction model construction,
a lot of time has been spent to study and comb.\\
In feature processing, there is also a feature of the commodity price that is not added
to the forecasting model.The main reason is that there are still some deficiencies 
in the treatment of price characteristics.\\
In the process of predictive analysis,
the theoretical and data support for feature analysis and model construction 
is not concise and powerful enough, which needs to be strengthened.\\
% In the process of predictive analysis,
% the theoretical and data support for feature analysis and model construction 
% is not concise and powerful enough, which needs to be strengthened.
% \begin{description}
%   \item[Problem Definition]
%   Formalize the problem of Group Outlying Aspects Mining by extending outlying aspects mining.

%   \item[GOAM algorithm]
%   Propose GOAM algorithm to solve the \emph{Group}\\
%   \emph{Outlying Aspects Mining} problem.

%   \item[Strategies]
%   Utilize the pruning strategies to \\ reduce time complexity.
% \end{description}
}
%%%%%%%%%% -------------------------------------------------------------------- %%%%%%%%%%


% Bottomblock
%%%%%%%%%% -------------------------------------------------------------------- %%%%%%%%%%
\colorlet{notebgcolor}{blue!20}
\colorlet{notefrcolor}{blue!20}
\note[targetoffsetx=8cm, targetoffsety=-4cm, angle=30, rotate=15,
radius=2cm, width=.26\textwidth]{
Acknowledgement
\begin{itemize}
    \item
    International Cooperation Project (Y7Z0511101)
    of IIE,
    Chinese Academy of Sciences
 \end{itemize}
}

%\note[targetoffsetx=8cm, targetoffsety=-10cm,rotate=0,angle=180,radius=8cm,width=.46\textwidth,innersep=.1cm]{
%Acknowledgement
%}

%\block[titlewidthscale=0.9, bodywidthscale=0.9]
%{Acknowledgement}{
%}
%%%%%%%%%% -------------------------------------------------------------------- %%%%%%%%%%

\end{columns}


%%%%%%%%%% -------------------------------------------------------------------- %%%%%%%%%%
%[titleleft, titleoffsetx=2em, titleoffsety=1em, bodyoffsetx=2em,%
%roundedcorners=10, linewidth=0mm, titlewidthscale=0.7,%
%bodywidthscale=0.9, titlecenter]

%\colorlet{noteframecolor}{blue!20}
\colorlet{notebgcolor}{blue!20}
\colorlet{notefrcolor}{blue!20}
\note[targetoffsetx=-13cm, targetoffsety=-12cm,rotate=0,angle=180,radius=8cm,width=.96\textwidth,innersep=.4cm]
{
\begin{minipage}{0.3\linewidth}
\centering
\includegraphics[width=24cm]{logos/tulip-wordmark.eps}
\end{minipage}
\begin{minipage}{0.7\linewidth}
{ \centering
 The $11^{th}$ International Conference on Knowledge Science,
  Engineering and Management (KSEM 2018),
  17-19/08/2018, Changchun, China
}
\end{minipage}
}
%%%%%%%%%% -------------------------------------------------------------------- %%%%%%%%%%


\end{document}

%\endinput
%%
%% End of file `tikzposter-template.tex'.
